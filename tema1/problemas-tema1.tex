\documentclass[11pt]{article}
\usepackage[spanish]{babel}
\usepackage[utf8]{inputenc}

% ------ Cargar estilo específico para la relación de problemas
\usepackage{problemas-MNII}

% ==============
\begin{document}
% ==============

\begin{problemas}
  \begin{problema}
    \begin{enumerate}
    \item El siguiente método iterativo, publicado en 1593 por
      François Viète, permite el cálculo de $\pi$ como producto
      infinito:
      \begin{align*}
        a_1 &= \sqrt{2}, \quad a_{k+1} = \sqrt{2+a_k}\\
        \frac{2}{\pi} & = \prod_{k\to\infty} \frac{a_i}2
      \end{align*}
    \end{enumerate}
  \end{problema}
  \begin{problema}
    El siguiente método iterativo, conocido como algoritmo de
    Gauss-Legendre o de Brent-Salamin, se utiliza para el cálculo de
    dígitos decimales de $\pi$:
    Fijados $a_0=1$, $b_0=1/\sqrt 2$, $t_0=1/4$, $p_0=1$, para cada entero
    $n\ge 0$,
    \begin{align*}
      a_{n+1}&=(a_{n}+b_{n})/2 , &\quad
      t_{n+1}&=t_n - p_n(a_n -a_{n+1})^2, \\
      b_{n+1} &=\sqrt{a_{n}\*b_{n}}, &\quad p_{n+1}&=2p_n,
    \end{align*}
    En cada iteración, se puede aproximar el valor de $\pi$ como: 
    $$
    \pi \approx x_n= \frac{(a_n+b_n)^2}{4t_n}.
    $$
  \end{problema}

  \begin{problema}
    Demostrar que en un método iterativo de orden $p$, el número de
    cifras exactas obtenidas en cada iteración (con respecto a la
    solución exacta, $\alpha$), se multiplica por $p$ en cada
    iteración.
    \begin{flushright}
      \scriptsize \textit{Indicación}: demostrar que si
      $d_k=-\log_{10}|x_k-\alpha|$, entonces $d_{n+1}=p d_n + r$, con
      $r=-\log_{10}\lambda$, siendo $\lambda$ la constante asintótica
      de error.
    \end{flushright}
    Para un método de orden uno y $\lambda=0.999$, ¿cuántas
    iteraciones serán necesarias para obtener una nueva cifra exacta?
    Para orden $p=1.01$, ¿cada cuántas iteraciones se dobla el número
    de cifras exactas? ¿Y para $p=1.1$? Demostrar que la sucesión
    $x_k=2^{-k}$ proporciona una aproximación de orden $1$ de
    $\alpha=0$ y determinar $\lambda$. Comprobar numéricamente que
    $d_{n+1}=p\, d_n + r$, para distintos valores de $n$.
  \end{problema}
\end{problemas}

\end{document}

%===============

%%% Local Variables: 
%%% mode: latex
%%% TeX-master: t
%%% End: 
