\documentclass[11pt]{article}
\usepackage[spanish]{babel}
\usepackage[utf8]{inputenc}

% ------ Cargar estilo específico para la relación de problemas
\usepackage{problemas-MNII}

% ==============
\begin{document}
% =============
\begin{flushright}
  \LARGE\it Relación de problemas. Tema \huge 1.\\
  \bigskip
\end{flushright}

\begin{problemas}

   \begin{problema}
     Nos proponemos encontrar la solución de la siguiente ecuación en
     el intervalo $[0,\pi]$: 
     $$
     \int_0^x (\sen s)^2 \, ds = 1.
     $$
   \end{problema}
   \begin{enumerate}
   \item Demostrar que el problema es equivalente a hallar un punto
     fijo de la función:
     $$
     g(x)=2 + \frac{\sen(2x)}{2}
     $$
     en el intervalo $[0,\pi]$.
   \item Comprobar que la función $g$ satisface las hipótesis del
     teorema del punto fijo en el intervalo $\displaystyle [1.6,2]\subset
     \bigg(\frac{\pi}{2},\frac{3\pi}{4}\bigg)$.
   \item Determinar cuantas iteraciones son necesarias para que el
     error cometido sea menor que $10^{-8}$. Calcular las dos primeras
     iteraciones.
   \end{enumerate}
  %  \begin{problema}
  %  Dada la ecuación $f(x)=e^x-(x-2)^2=0$, demostrar que sólo posee una
  %  raíz real.
  %  \begin{flushright}\em\small
  %    Idea: probar que $f'(x)>0$ para todo $x\in\Rset$. Para ello,
  %    cuando $x>2$, aplicar a $e^x$ el teorema del valor medio en
  %    $[2,x]$ para probar que $e^x>2(x-2)$.
  %  \end{flushright}
  % \end{problema}
  
  \end{problemas}
\end{document}

%%% Local Variables: 
%%% mode: latex
%%% TeX-master: t
%%% End: 
