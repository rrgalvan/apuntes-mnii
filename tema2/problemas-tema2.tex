\documentclass[11pt]{article}
\usepackage[spanish]{babel}
\usepackage[utf8]{inputenc}

% ------ Cargar estilo específico para la relación de problemas
\usepackage{problemas-MNII}



% ==============
\begin{document}
% =============

\begin{center}
  \large \textbf{Relación de poblemas} \\ [0.2em]
  \Large \textit{Tema 2. Interpolación y aproximación de funciones}
  % \\
  % \rule{\linewidth}{0.2pt}
\end{center}

\medskip

\begin{problemas}
  \begin{problema}
    Utilizar el método de bisección para aproximar las raíces de las
    siguientes ecuaciones con una tolerancia menor que
    $\varepsilon=10^{-8}$.
    \begin{enumerate}
    \item $e^x-3x^2=0$.
    \item $x^3=x^2+x+1$.
    \end{enumerate}
  \end{problema}

  \begin{problema}
    Obtener una cota del error cuando interpolamos la función
    $f(x)=e^{2-x}$ usando $n$
    nodos en $[-2,2]$. Estudiar gráficamente la evolución del error
    cuando con $n=5,\dots,15$.
  \end{problema}

\end{problemas}

\end{document}

%===============

%%% Local Variables: 
%%% mode: latex
%%% TeX-master: t
%%% End: 
