
\chapter[Raíces de funciones de una variable]{Interpolación y aproximación de funciones%
  \footnote{\licenseInfo}}
\label{cha:Interpolacion-aproximacion}

El problema de la interpolación surge cuando se intenta construir una
función (a la que suele llamarse función interpolante) de la que
solamente conocemos sus valores en una serie de puntos (que suelen
llamarse nodos de interpolación).  Este tipo de problemas surge, por
ejemplo, si disponemos de un cierto número de datos puntos obtenidos
por muestreo o a partir de un experimento y pretendemos construir una
función que los ajuste, o bien si pretendemos aumentar el tamaño de
una fotografía, rellenando la imagen inicial con datos que son
<<inventados>>.

Un problema estrechamente ligado con el de la interpolación es la
aproximación de una función complicada por una más simple y por tanto
más fácil de manejar (derivar, integrar, etc). Por ejemplo, los
métodos usuales para la aproximación de la integral de una función en
un intervalo se basan en sustituirla por una función polinómica.
En lo que sigue, supondremos que la función interpolante es un
polinomio (aunque existen otras posibilidades: interpolación por
funciones racionales, trigonométricas, exponenciales, etc.), puesto
que son las funciones más sencillas de manejar y constituyen la
base de los métodos numéricos que estudiaremos en los próximos capítulos. 

\section{Interpolación de Lagrange}
\label{sec:interp-de-lagrange}

Consideremos conjunto de datos definido por un soporte de $n+1$ nodos
distintos, $S=\{x_0,x_1,\dots,x_n\}\subset \Rset$, junto a $n+1$
puntos $y_i \in \Rset$:
\begin{equation}
  \begin{array}{r|lllcl}
    \toprule
    x & x_0 & x_1 & x_2 & \dots & x_n
    \\ \midrule
    y & y_0 & y_1 & y_2 & \dots & y_n
    \\
    \bottomrule
  \end{array}
  \label{eq:tabla-datos-lagrange}
\end{equation}
y denotemos por $\Pol_n[x]$ al conjunto de los polinomios de grado
menor o igual a $n$ en la variable $x$. Planteamos el problema de la
interpolación de Lagrange como:
\begin{equation}
  \tag{P$_{IL}$}
  \text{Hallar } p\in\Pol_n[x] \tq p(x_i)=y_i \quad \forall i=0,1,\dots,n.
  \label{eq:problema-interpol-lagrange}
\end{equation}
\begin{definition}
  \label{def:interpolador-lagrange}
  Decimos que un polinomio $p$ interpola el conjunto de
  datos~\eqref{eq:tabla-datos-lagrange} si $p$ es solución del
  problema~\eqref{eq:problema-interpol-lagrange}. En el caso en que
  los datos vengan dados por una función, $y_i=f(x_i)$ decimos que $p$
  es el polinomio de interpolación de Lagrange (o interpolador de
  Lagrange) de la función $f$ en los nodos $x_0$, $x_1$, \dots, $x_n$.
\end{definition} 
En adelante supondremos $f\in C^0([a,b])$ para cierto intervalo
$[a,b]\subset\Rset$ que contiene a los nodos de interpolación,
$x_0,x_1,\dots,x_n \in [a,b]$.  El resto de esta sección se estructura
de la siguiente forma:
\begin{enumerate}
\item Estudio de la existencia y unicidad de solución del
  problema~\eqref{eq:problema-interpol-lagrange}.
\item Deducción de algoritmos para el cálculo efectivo del polinomio
  de interpolación.
\item Análisis del error de interpolación, es decir de la diferencia
  $|f(x)-p(x)|$ con $x\in [a,b]$.
\end{enumerate}

\subsection{Existencia y unicidad del interpolador de Lagrange}
\label{sec:exist-y-unic-lagrange}

\begin{theorem}
  \label{thm:existencia-unicidad-lagrange}
  Dado un soporte de $n+1$ nodos distintos $x_0$, $x_1$,\dots, $x_n
  \in \Rset$ y dados $y_0$, $y_1$,\dots, $y_n\in\Rset$ cualesquiera,
  existe una única solución de~\eqref{eq:problema-interpol-lagrange},
  es decir existe un único polinomio $p\in\Pol_n[x]$ tal que
  $p(x_i)=y_i$ para todo $i=0,1,\dots,n$.
\end{theorem}
\begin{proof}~
\etapa{Etapa a) Equivalencia con un sistema lineal de ecuaciones.}
  Todo polinomio $p\in\Pol_n[x]$, está unívocamente determinado por
  $n+1$ coeficientes, $a_0$, $a_1$, \dots, $a_n\in\Rset$ tales que
  \begin{equation}
    p(x)=a_0 + a_1 x + \cdots + a_n x^n.
  \end{equation}
  A su vez, el polinomio de interpolación se caracteriza por verificar
  $n+1$ ecuaciones,
  $$
  p(x_i)=y_i, \quad i=0,1,\dots,n,
  $$
  que se pueden escribir como el siguiente sistema de $n+1$ ecuaciones
  con $n+1$ incógnitas (cuya matriz cuadrada, $A$, se conoce como
  matriz de Vandermonde):
  \begin{equation}
    \begin{pmatrix}
      1 & x_0& \cdots & x_0^n \\
      1 & x_1& \cdots & x_1^n \\
      \vdots & \vdots & & \vdots \\
      1 & x_n& \cdots & x_n^n 
    \end{pmatrix}
    \begin{pmatrix}
      a_0 \\ a_1 \\ \vdots \\ a_n
    \end{pmatrix}
    =
    \begin{pmatrix}
      y_0 \\ y_1 \\ \vdots \\ y_n
    \end{pmatrix}.
  \end{equation}
  Por tanto la existencia y unicidad del polinomio de interpolación es
  equivalente a la existencia de unos únicos coeficientes
  $(a_0,a_1,...,a_n)$ que solucionan el sistema anterior.
  
  \etapa{Etapa b) Existencia y unicidad de solución.}  Veremos que
  este sistema tiene una única solución. Para ello, como $A$ es una
  matriz cuadrada, basta ver que $|A|\neq 0$, para lo que es suficiente
  probar que el sistema homogéneo $Au=0$, con $u\in\Rset^{n+1}$, tiene
  como única solución $u=0$.

  Supongamos que $u=(a_0,a_1,\dots,a_n) \in\Rset^{n+1}$ verifica
  $Au=0$. Entonces, el polinomio asociado, $p(x)=a_0 + a_1x + \cdots
  + a_n x^n$, verifica $p(x_i)=0$ para $i=0,1,\dots,n$.  Por lo tanto $p$ es un
  polinomio de grado $n$ con $n+1$ raíces distintas, luego
  $p=0$, es decir $u=(a_0,a_1,\dots,a_n)=0$. 
\end{proof}

\begin{remark}
  \label{rk:1}
  En la demostración anterior podríamos haber probado directamente
  que el determinante de la matriz de Vandermonde es distinto de cero,
  pues los puntos $x_0$, $x_1$, \dots, $x_n$ son distintos
  entre sí. Pero se ha optado por realizar la etapa b) debido a que el
  argumento empleado (es decir, en un sistema lineal con matriz
  cuadrada, la unicidad [o sea $Au=0 \Rightarrow u=0$] implica la existencia)
  es generalizable a otras demostraciones de existencia y unicidad que
  aparecerán, por ejemplo, en la siguiente sección.
\end{remark}

\subsection{Construcción del polinomio de interpolación de Lagrange}
\label{sec:construcion--polinomio-lagrange}

De la demostración del Teorema~\ref{thm:existencia-unicidad-lagrange}
se puede deducir un primer método para la construcción del
interpolador de Lagrange: la resolución del sistema lineal con matriz
de Vandermonde. Sin embargo, este método es costoso, pues se trata de
una matriz llena (con pocos ceros) y además se puede demostrar que
esta matriz tiene un mal condicionamiento cuando $n$ crece hacia
infinito. Estudiaremos dos métodos más adecuados.

\subsubsection{Fórmula de Lagrange}
Dado un soporte de $n+1$ puntos distintos, $S=\{x_0,x_1,\dots,x_n\}$,
existe un único polinomio $L_i\in \Pol_n[x]$, que interpola los
valores $y_0=0$, \dots, $y_{i-1}=0$, $y_i=1$, $y_{i+1}=0$, \dots,
$y_n=0$ (debido al Teorema~\ref{thm:existencia-unicidad-lagrange}). Es
decir, $L_i$ verifica (para cada $x_j$):
\begin{equation}
  L_i(x_j)= \delta_{ij} = 
  \left\{\begin{array}{l}
      1 \text{ si } i=j, \\\noalign{\medskip} 0 \text{ si } i\neq j
    \end{array}\right. \quad \text{(función delta de Kronecker).}
\end{equation}
En concreto, es fácil comprobar que el polinomio $L_i(x)$ viene dado
por el siguiente polinomio (de orden exactamente igual a $n$):
\begin{equation}
  L_i(x)=
  \begin{array}{c@{}c@{}c@{}c@{}c@{}c@{}c}
    (x-x_0) & \cdots & (x-x_{i-1}) & \cdot & (x-x_{i+1}) & \cdots &
    (x-x_n)
  \\ \hline
    (x_i-x_0) & \cdots & (x_i-x_{i-1}) & \cdot & (x_i-x_{i+1}) & \cdots &
    (x_i-x_n)
  \end{array}
%   \frac{x-x_0}{x_i-x_0}\cdots
%   \frac{x-x_{i-1}}{x_i-x_{i-1}}\cdot
%   \frac{x-x_{i+1}}{x_i-x_{i+1}}\cdots
%   \frac{x-x_n}{x_i-x_n}.
\end{equation}
El conjunto de los polinomios $\{ L_0, L_1,\dots, L_n \}$ se llama
base de Lagrange asociada al soporte $S=\{x_0,x_1,\dots,x_n\}$. 

Así, dado un conjunto de valores $\{y_0,y_1,\dots,y_n\}$, el polinomio
que los interpola en el soporte $S$ viene dado por la siguiente
combinación lineal:
\begin{equation}
  p(x)= y_0L_0(x) + y_1 L_1(x) + \cdots + y_n L_n(x),
\end{equation}
pues, por construcción de los polinomios $L_i$, el polinomio anterior
es solución de~(\ref{eq:problema-interpol-lagrange}).

\begin{example}
  \label{sec:formula-de-lagrange}
  Calculemos el polinomio de interpolación de Lagrange asociado a
  los puntos:
  \begin{align*}
    (x_0, y_0)&=(-1,0), &\quad (x_2, y_2)&=(1,2),\\ 
    (x_1, y_1)&=(0,2),   &\quad (x_3, y_3)&=(2,6).
  \end{align*}
  La base de Lagrange asociada al soporte
  $S=\{x_0,x_1,x_2,x_3\}=\{-1,0,1,2\}$ es:
  \begin{small}
    \begin{align*}
      L_0(x) & =
      \frac{(x-x_1)(x-x_2)(x-x_3)}{(x_0-x_1)(x_0-x_2)(x_0-x_3)} =
      \frac{(x-0)(x-1)(x-2)}{(-1-0)(-1-1)(-1-2)} =
      -{{\left(x-2\right)\,\left(x-1\right)\,x}\over{6}}, \\
      \noalign{\medskip} L_1(x) & =
      \frac{(x-x_0)(x-x_2)(x-x_3)}{(x_1-x_0)(x_1-x_2)(x_1-x_3)} =
      \frac{(x+1)(x-1)(x-2)}{(0+1)(0-1)(0-2)} =
      {{\left(x-2\right)\,\left(x-1\right)\,\left(x+1\right)}\over{2}},
      \\ \noalign{\medskip} L_2(x) & =
      \frac{(x-x_0)(x-x_1)(x-x_3)}{(x_2-x_0)(x_2-x_1)(x_2-x_3)} =
      \frac{(x+1)(x-0)(x-2)}{(1+1)(1-0)(1-2)} =
      -{{\left(x-2\right)\,x\,\left(x+1\right)}\over{2}}, \\
      \noalign{\medskip} L_3(x) & =
      \frac{(x-x_0)(x-x_1)(x-x_2)}{(x_3-x_0)(x_3-x_1)(x_3-x_2)} =
      \frac{(x+1)(x-0)(x-1)}{(2+1)(2-0)(2-1)} =
      {{\left(x-1\right)\,x\,\left(x+1\right)}\over{6}}.
    \end{align*}
  \end{small} 
  Así tenemos el polinomio de interpolación para los valores
  $\{y_0,y_1,y_2,y_3\}=\{0,2,2,6\}$:
  \begin{align*}
    p(x)&= \sum_{i=0}^3 y_iL_i(x) 
    = 0\cdot L_0(x) + 2\cdot L_1(x)+ 2\cdot L_2(x)+6\cdot L_3(x)
    \\ &=(x-1)x(x+1)-(x-2)x(x+1)+(x-2)(x-1)(x+1)
    =x^3-x^2+2.
  \end{align*}
  Podemos comprobar que no hubo errores, pues efectivamente $p(x)$ es
  un polinomio que interpola los valores
  $(x_i,y_i)$ (el único de $\Pol_3[x]$ que lo hace):
  \begin{align*}
    p(x_0)&=p(-1)=(-1)^3-(-1)^2+2=0=y_0, & p(x_2)&=1^3-1^2+2 = 2=y_2, \\
    p(x_1)&=p(0) = 2=y_1, & p(x_3)&=2^3-2^2+2 = 6=y_3.
  \end{align*}
\end{example}

\begin{remark}
  Una ventaja de la fórmula de Lagrange es que, fijado un soporte $S$
  y una vez calculados los polinomios $\{L_0,L_1,\dots, L_n\}$ que
  forman a la base de Lagrange asociada, es muy sencillo calcular los
  polinomios que interpolan tantos conjuntos de valores,
  $\{y_0,y_1,\dots,y_n\}$ como deseemos. 
  
  Sin embargo, su inconveniente es que si deseamos añadir un punto más
  al soporte (y por tanto realizar la interpolación con un polinomio
  de un grado mayor) es necesario volver a calcular toda la base de
  polinomios de Lagrange.
\end{remark}

\subsubsection{Fórmula de Newton}
\label{sec:formula-de-newton}
A continuación, dado el soporte $S=\{x_0,x_1,\dots,x_n\}$ de $n+1$ puntos
distintos y dados $n+1$ valores $\{y_0,y_1,\dots,y_n\}$, sea
$p_n$ el (único) polinomio $p_n \in \Pol_n[x]$, que
interpola los valores $\{y_i\}_{i=0}^n$, es decir
\begin{equation}
p_n(x_i)=y_i, \quad \forall i=0,...,n.
\label{eq:interpol.1}
\end{equation}
Expresaremos este polinomio de la siguiente forma:
\begin{align}
  \notag
  p_n(x)&=a_0 + a_1 (x-x_0) + a_2(x-x_0)(x-x_1) + \cdots 
  + a_n(x-x_0)(x-x_1)\cdots(x-x_{n-1}) \\
  &=\sum_{k=0}^n a_k \prod_{i=0}^{k-1}(x-x_i),
\label{eq:pol.newton.dif-div}
\end{align}
donde $a_0$, $a_1$, \dots, $a_n\in\Rset$ son coeficientes que se
pueden ajustar para que se verifiquen las
ecuaciones~(\ref{eq:interpol.1}):
\begin{align*}
  p_n(x_0)&=y_0 \Rightarrow a_0=y_0.\\
  p_n(x_1)&=y_1 \Rightarrow a_0+a_1(x_1-x_0)=y_1 
  \Rightarrow a_1=\frac{y_1-y_0}{x_1-x_0}.\\
  p_n(x_2)&=y_2 \Rightarrow a_0+a_1(x_2-x_0)+a_2(x_2-x_0)(x_2-x_1)=y_2 
  \Rightarrow a_2=\cdots\\
  &\vdots
\end{align*}
Estos coeficientes se pueden calcular mediante una regla de
recurrencia sencilla, conocida como fórmula de las diferencias dividas
de Newton, que veremos a continuación. Históricamente se ha utilizado
la siguiente notación (en la que se asume que los datos vienen dados
por cierta función, $y_k=f(x_k)$, y se expresa el hecho de que cada
$a_k$ depende solamente de los valores de $f$ en
$\{x_0,x_1,\dots,x_k\}$):
% De forma más compacta, podemos escribir:
% \begin{equation}
%   p_n(x)=\sum_{k=0}^n a_k \prod_{i=0}^{k-1}(x-x_i).
%   \label{eq:interpol.2}
% \end{equation}
\begin{definition}
  Dada una función $f$ de la cual se conocen los valores $y_0=f(x_0)$,
  $y_1=f(x_1)$, \dots, $y_n=f(x_n)$, 
  a cada uno de los coeficientes, $a_k$, del polinomio $p_n(x)$
  anterior se le llama \resaltar{diferencia
    dividida} (de orden $k$) de $f$ en los nodos $x_0$, $x_1$, \dots
  $x_n$. La diferencia dividida de orden $k$ se denota como
  $$
  a_k = f[x_0,x_1,\dots,x_k].
  $$
\end{definition}

El siguiente lema expresa las propiedad más importante de las
diferencias divididas%
\footnote{El lector interesado en estudiar más a fondo las
  diferencias divididas  puede consultar por ejemplo la
  siguiente referencia Bibliográfica: Kincaid, D., Cheney, W.,
  Análisis Numérico. \textit{Las matemáticas del cálculo
    científico}. Addison-Wesley-Iberoamericana 1994.}.
% \begin{lemma}
% \label{lem:interpol1}
% El polinomio de interpolación de los datos $y_i=f(x_i)$ se puede
% escribir como
% $$
% p(x) = a_0 + a_1x + a_2x^2 + \cdots a_nx^n,
% $$
% donde cada $a_k=f[x_0,...,x_k]$.
% \end{lemma}
% \begin{proof}
%   Consideremos el polinomio de interpolación, para los valores
%   $y_i=f(x_i)$, $i=0,...,n$, que hemos expresado de la
%   forma~(\ref{eq:pol.newton.dif-div}). Por definición, cada
%   $a_k=f[x_0,...,x_k]$.

%   Si observamos último sumando de tenemos que
%   $$
%   (x-x_0)(x-x_1)\cdots (x-x_{n-1}) = x^n + \text{términos de grado
%     menor que $n$},
%   $$
%   por lo tanto $a_n=f[x_0,...,x_n]$ es el coeficiente de $x^n$ en el
%   polinomio de interpolación.

%   Consideremos ahora el polinomio resultante de truncar el último sumando:
%   \begin{align*}
%     p_{n-1}(x)= \sum_{k=0}^{n-1} a_k \prod_{i=0}^{k-1}(x-x_i).
%   \end{align*}
%   Es fácil ver que $p_{n-1}$ es el único polinomio de grado $n-1$ que
%   interpola los valores $(x_i,y_i)$ para $i=0,...,n-1$. En concreto,
%   %\begin{equation*}
%    $  p_{n-1}(x)  =p_n(x)-a_n \prod_{i=0}^{n-1}(x-x_i),$
%   % \end{equation*}
%    luego $p_{n-1}(x_i)=y_i$ para $i=0,\dots,n-1$, es decir $p_{n-1}$
%    es el (único) polinomio de interpolación de estos datos.
   
%    Si miramos el último sumando de $p_{n-1}(x)$, tenemos 
%   $$
%   (x-x_0)(x-x_1)\cdots (x-x_{n-2}) = x^{n-1} + \text{términos de grado
%     menor que $n-1$},
%   $$
% %
% \end{proof}
% 
 
\begin{lemma}
  \label{lem:1}
  Las formulas divididas satisfacen la siguiente relación recursiva:
  \begin{equation}
    \begin{aligned}
      f[x_i]&=y_i, \quad i=0,...n,
      \\
      f[x_0,x_1,\dots,x_n]&=
      \frac{f[x_1,x_2,\dots,x_n]-f[x_0,x_1,\dots,x_{n-1}]}{x_n - x_0}.
    \end{aligned}
    \label{eq:interpol.dif-div.recursiva}
  \end{equation}
\end{lemma}
En~(\ref{eq:interpol.dif-div.recursiva}), los puntos
$x_0,...,x_n$ deben interpretarse como variables, en un sentido amplio
(pues de hecho $\{x_k\}_{k=0}^n$ son puntos genéricos, con $n$
cualquiera, y podrían volver a enumerarse de cualquier otra forma, por
ejemplo $\{x_k\}_{k=i}^{i+j}$). Por lo tanto, son ciertas todas las
ecuaciones del tipo
\begin{equation*}
  f[x_i,x_{i+1},\dots,x_{i+j}]=
  \frac{f[x_{i+1},\dots,x_{i+j}]-f[x_i,\dots,x_{i+j-1}]}{x_j - x_i}.
\end{equation*}

Aunque la demostración del lema anterior es de carácter técnico, la
incluimos de modo ilustrativo.
\begin{proof}~ 
  \etapa{Etapa 1.} Sean $p_n\in\Pol_n[x]$ y $p_{n-1}\in
  \Pol_{n-1}[x]$ los polinomios de interpolación de los valores
  $\{x_i,f(x_i)\}_{i=0}^n$ y $\{x_i,f(x_i)\}_{i=0}^{n-1}$,
  respectivamente. Sea $q_{n-1}\in\Pol_{n-1}[x]$ el polinomio de
  interpolación de $\{x_i,f(x_i)\}_{i=1}^{n}$.

  Entonces, se verifica:
  \begin{equation}
    p_n(x)=q_{n-1}(x)+\frac{x-x_n}{x_n-x_0}\left[q_{n-1}(x)-p_{n-1}(x)\right].\label{eq:1}
  \end{equation}
  Esto es debido a que (como es fácil comprobar usando la definición
  de $p_{n-1}$ y $q_{n-1}$) el segundo miembro es un polinomio de
  grado $n$ que interpola los valores $\{x_i,f(x_i)\}_{i=0}^n$. Luego,
  por unicidad del polinomio de interpolación, tiene que conicidir con
  $p_n$. 

  \etapa{Etapa 2.}  Por definición, $a_n=f[x_0,...,x_n]$ es el
  coeficiente del término $ (x-x_0)(x-x_1)\cdots (x-x_{n-1}) $ en la
  expresión~(\ref{eq:pol.newton.dif-div}) del polinomio de
  interpolación. Pero teniendo en cuenta que
   $$
   (x-x_0)(x-x_1)\cdots (x-x_{n-1}) = x^n + \text{términos de grado
     menor que $n$},
   $$
   $a_n=f[x_0,...,x_n]$ coincide con el coeficiente de $x^n$ en el
   polinomio de interpolación. En general, para todo $j<k$, $f[x_j,...,x_{k}]$ es el
   coeficiente del término de mayor grado en el polinomio que
   interpola los valores $\{x_i,f(x_i)\}_{i=j}^{k}$.
 
   \etapa{Etapa 3.} A continuación, examinamos los coeficientes del
   término de mayor grado, es decir de $x^n$, en el primer y el
   segundo miembro de~(\ref{eq:1}). Estos coeficientes deben ser
   iguales, lo que nos lleva inmediatamente a la fórmula
   recursiva~(\ref{eq:interpol.dif-div.recursiva}) que buscamos (en el
   segundo miembro de~(\ref{eq:1}) sólo influye el término de mayor
   grado de $q_{n-1}(x)-p_{n-1}(x)$).
\end{proof}

El lema~\ref{lem:1} nos permite organizar el cálculo de las
diferencias divididas a través de un procedimiento sencillo
\begin{center}
  \includegraphics[width=0.9\linewidth]{tema2/diferencias-div}
\end{center}

\begin{example}
  Calcular el polinomio de interpolación de Lagrange de $f(x)=2^x$ en
  $S=\{-2,-1,0,1,2\}$. 
\end{example}

Una fórmula poco costosa computacionalmente para la evaluación del
polinomio de interpolación en un punto es la \resaltar{fórmula de
  Ruffini-Horner}: 
\begin{equation*}
  p_n(r) = a_0 + (r-x_0)(a_1 + (r-x_1)(a_2 + \cdots + (r-x_n)a_n)\cdots)
\end{equation*}
En esta fórmula, el número de multiplicaciones se ve reducida
sustancialmente con con respecto a la
expresión~(\ref{eq:pol.newton.dif-div}) .
\begin{example}
  Utilizaremos la interpolación de $f(x)=2^x$ y la fórmula de
  Ruffini-Horner para obtener una aproximación de $\sqrt{2}$.
\end{example}

\subsection{Expresión del error}
\label{sec:error-interpol-lagrange}
Como antes, consideremos una función $f(x)$ definida en un intervalo
$[a,b]$ de la que conocemos los valores $y_i=f(x_i)$ en un soporte
$\{x_0,x_1,\dots,x_n\}$. El siguiente resultado nos ofrece, suponiendo
regularidad en $f$, una cota del error de interpolación en cada punto
de $[a,b]$:
\begin{theorem}
  \label{thm:error-interpol-lagrange}
  Sea $f\in C^{n+1}([a,b])$ y fijemos un soporte de $n+1$ puntos
  distintos, $S=\{x_0,x_1,\dots,x_n\}\subset[a,b]$. Sea
  $p_n\in\Pol_n[x]$ el polinomio de interpolación asociado a los valores
  $y_i=f(x_i)$, $i=0,...,n$. Para cada $x\in[a,b]$, existe $\xi_x\in
  (a,b)$ tal que:
  \begin{equation}
    \label{eq:expresion-error-interpol}
    f(x)-p_n(x)=\frac{f^{n+1}(\xi_x)}{(n+1)!} (x-x_0)(x-x_1)\dots(x-x_n).
  \end{equation}
\end{theorem}

\begin{proof}
  Si $x=x_i$ para algún $i=0,...,n$,
  entonces~(\ref{eq:expresion-error-interpol}) es evidente (ambos
  miembros de la igualdad se anulan).

  \etapa{Etapa 1.}
  Fijemos, en lo que sigue, $x\neq x_i$ para todo $i=0,...,n$. Denotando
  \begin{equation}
  \omega_S=(x-x_0)(x-x_1)\dots (x-x_n),
 \label{eq:w_S:3} 
\end{equation}
  definamos la siguiente función auxiliar:
  \begin{equation}
    \phi(t)=f(t)-p_n(t)-\lambda_x\, \omega_S, \quad \forall t\in[a,b],
  \label{eq:interpol:2}
\end{equation}
  donde $\lambda_x$ se calcula de forma que, $\phi(x)=0$. Es
  decir,
  $$
  \lambda_x = \frac{f(x)-p_n(x)}{\omega_S(x)}.
  $$
  Si conseguimos demostrar que existe $\xi_x\in (a,b)$ tal que
  $\phi^{n+1)}(\xi_x)=0$ hemos terminado, pues derivando
  en~(\ref{eq:interpol:2}):
  \begin{align*}
    0=\phi^{n+1)}(\xi_x)&=f^{n+1)}(\xi_x)-p_n^{n+1)}(\xi_x)-\lambda_x\cdot\omega_S^{n+1)}(\xi_x)
    \\
    &=f^{n+1)}(\xi_x)- \frac{f(x)-p_n(x)}{\omega_S(x)}\cdot (n+1)!,
  \end{align*}
  de donde podemos despejar $f(x)-p_n(x)$ para
  obtener~(\ref{eq:expresion-error-interpol}) \etapa{Etapa 2.}
  Utilizando repetidas veces el teorema de Rolle podemos comprobar que
  existe $\xi_x\in (a,b)$ tal que $\phi^{n+1)}(\xi_x)=0$:

  Debido a su definición, $\phi$ verifica:
  $\phi(x_0)=\phi(x_1)=\cdots=\phi(x_n)=0$ y además, por definición de
  $\lambda_x$, $\phi(x)=0$. Como $\phi\in C^{n+1}([a,b])$ y se anula
  en $n+2$ puntos distintos, por el Teorema de Rolle tenemos que
  $\phi'(x)$ se anula en $n+1$ puntos distintos de $(a,b)$. Aplicando
  sucesivamente el Teorema de Rolle, llegamos a que $\phi^{n+1)}\in
  C^0([a,b])$ y se anula (al menos) en un punto, $\xi_x\in (a,b)$.
\end{proof}

\begin{remark}[Cota uniforme del error]
  \label{rk:2}
  El teorema anterior nos proporciona la siguiente cota uniforme del
  error:
  \begin{equation}
    ||f-p_n||_{\infty} \le \frac{M_{n+1}}{(n+1)!}||\omega_S||_\infty,
    \label{eq:cota-error-interpol-1}
  \end{equation}
  donde para cualquier función $g\in C^0([a,b])$ denotamos
  \begin{equation*}
    ||g||_\infty = \max_{x\in[a,b]} |g(x)|  \quad (\text{norma
      uniforme de $g$})
  \end{equation*}
  y $M_{n+1}=||f^{n+1)}||_\infty$, mientras que $\omega_S$ está
  definido en~(\ref{eq:w_S:3}).
  En particular, se tiene que:
  \begin{equation}
    ||f-p_n||_{\infty} \le \frac{M_{n+1}}{(n+1)!}(b-a)^{n+1}.
    \label{eq:cota-error-interpol-2}
  \end{equation}
\end{remark}

\begin{remark}[Convergencia uniforme de los polinomios de interpolación]
  A la vista de la cota uniforme~(\ref{eq:cota-error-interpol-1}), es
  natural preguntarse si $||f-p_n||_{\infty}$ converge a cero, es
  decir si
\end{remark}
%%% Local Variables:
%%% mode: latex
%%% TeX-master: "../apuntes-MNII.tex"
%%% End: 
