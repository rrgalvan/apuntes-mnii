\documentclass[11pt]{article}
\usepackage[spanish]{babel}
\usepackage[utf8]{inputenc}

% ------ Cargar estilo específico para la relación de problemas
\usepackage{problemas-MNII}

% ==============
\begin{document}
% =============
\begin{flushright}
  \LARGE\it Relación de problemas. Tema \huge 3.\\
  \bigskip
\end{flushright}

\begin{problemas}

  \begin{problema}
    Deducir la fórmula compuesta de los trapecios, así como su
    expresión del error. Utilizando esta fórmula, calcular una
    aproximación de la integral
    $$
    \int_0^1 \log(1+x^2)\, dx
    $$
    en la que podamos garantizar que el error de cuadratura es menor que
    $\varepsilon=0.001$. ¿En cuántos subintervalos hay que dividir el
    intervalo $[0,1]$?
  \item Se considera la fórmula de cuadratura
    \begin{equation*}
      \int_0^1 f(x)\,dx \approx c_0 f(0) + c_1 f\bigg(\frac{1}{3}\bigg)
      + c_2 f\bigg(\frac{2}{3}\bigg) + c_3 f(1).
    \end{equation*}
    \begin{itemize}
    \item Calcular los coeficientes $c_0$, $c_1$, $c_2$ y $c_3$ para
      los que la fórmula de cuadratura es exacta para polinomios de
      grado menor o igual que $3$. Comprobar que la fórmula resultante
      no proporcionan el valor exacto para polinomios de orden $4$.
    \item Aplicar esta fórmula de cuadratura para aproximar la
      siguiente integral:
      \begin{equation*}
        \int_0^1 \big(-\log(\cos x) \big)\,dx
      \end{equation*}
    \item Realizar el cambio de variables $t=(x-a)/(b-a)$ para
      transformar la integral
      \begin{equation*}
        \int_a^b f(x)\, dx
      \end{equation*}
      en una integral sobre el intervalo $[0,1]$. Utilizar la fórmula
      de cuadratura del apartado anterior para deducir una 
      regla de cuadratura en cualquier intervalo $[a,b]$.
    \item Sabiendo que la regla anterior es la fórmula de cuadratura
      de Newton-Côtes cerrada con $4$ nodos, obtener una expresión que
      de la fórmula compuesta correspondiente, que permita aplicarla.
    \item Aplicar la fórmula compuesta con $20$ subintervalos para
      aproximar la siguiente integral:
      \begin{equation*}
        \int_0^{10} \log\big(1+3e^{-x^2}\big)\,dx.
      \end{equation*}
    \end{itemize}
  \end{problema}
  
  \begin{problema}
  Determinar las constantes $A_0$, $A_1$ y $A_2$ de modo que la
  fórmula de cuadratura
  \begin{equation*}
    \int_0^h f(x)\,dx = h \bigg\{
    A_0f(0) + A_1f\bigg(\frac{h}{3}\bigg) + A_2 f(h) \bigg\}
  \end{equation*}
  sea exacta para polinomios del mayor grado posible. Utilizarla para
  aproximar la integral
  \begin{equation*}
    \int_0^{1/2} e^{-x^2}\, dx.
  \end{equation*}
  \end{problema}
 
  \begin{problema}
    Determinar los coeficientes $\alpha_0$, $\alpha_1$,  $\beta_0$ y
    $\beta_1$ para los que la fórmula de cuadratura
    \begin{equation*}
      \int_a^b f(x)\, dx \approx \alpha_0 f(a) + \alpha_1 f(b)
      + \beta_0 f'(a) + \beta_1 f'(b)
    \end{equation*}
    tiene el mayor orden posible. ¿Cuál es este orden de precisión?.
    Sabiendo que el error puede escribirse como
    \begin{equation*}
      \gamma (b-a)^5 f^{4)}(\xi),
    \end{equation*}
    para algún $\xi\in(a,b)$, determinar el valor de $\gamma$.
  \end{problema}

  \begin{problema}
    Para calcular un valor aproximado de $\ln 2$, se utilizará el
    hecho de que la integral
    \begin{equation*}
      \int_1^2 \frac{1}{x}\, dx
    \end{equation*}
    concide con este valor. Averiguar el número de subintervalos
    necesarios para que, al usar la fórmula compuesta de Simpson, el
    error sea inferior a $10^{-3}$. Utilizando el ordenador, calcular
    una aproximación de $\log 2$ y comprobar el error.
  \end{problema}

  \begin{problema}
    Se considera la fórmula de cuadratura
    \begin{equation*}
      \int_0^1 f(x)\, dx \approx A (f(x_0) + f(x_1)).
    \end{equation*}
    \begin{enumerate}
    \item Hallar el valor de $A$, $x_0$ y $x_1$ para que la
      fórmula sea exacta para polinomios del mayor grado
      posible. ¿Cuál es éste?
    \item Comprobar, mediante el cambio de variables $t=2x-1$, que
      la fórmula de cuadratura coincide con la regla de Gauss con
      dos nodos:
      \begin{equation*}
        \int_{-1}^1 g(x)\,dx \approx g\bigg(\frac{-1}{\sqrt
          3}\bigg) + g\bigg(\frac{1}{\sqrt 3}\bigg).
      \end{equation*}
    \end{enumerate}
  \end{problema}
  
  \begin{problema}
    La regla de Lobatto es una fórmula de cuadratura gaussiana
    definida en $[-1,1]$ de la siguiente forma:
    \begin{equation*}
      \int_{-1}^1 f(x)\, dx = A_0 f(-1) + \sum_{i=1}^{n-1} A_if(x_i) +
      A_nf(1).
    \end{equation*}
    \begin{itemize}
    \item Detallar la expresión de la regla de Lobatto para
      $n=3$. ¿Cuál es el orden de precisión de esta regla?
    \item Realizando un cambio de variable adecuado, utilizar la regla
      de Lobatto (con $n=3$) para estimar
      \begin{equation*}
        \int_1^2 e^x\, dx.
      \end{equation*}
    \end{itemize}
  \end{problema}
\end{problemas}

\end{document}

% ===============
=======

\begin{problemas}
  \begin{problema}
    Demostrar que la función $f(x)=x^2e^{x}-1$ tiene un único
    cero. Utilizar el método de bisección para aproximarlo con un
    error menor que $\varepsilon=10^{-7}$. Comparar el resultado con
    la solución aproximada que proporciona la función \texttt{fsolve}
    (contenida en el paquete \texttt{optimize} de \textit{Scipy}).
  \end{problema}
  \begin{problema}
    Utilizar el método de bisección para aproximar las raíces de las
    siguientes ecuaciones con una tolerancia menor que
    $\varepsilon=10^{-8}$.
    \begin{enumerate}
    \item $e^x-3x^2=0$.
    \item $x^3=x^2+x+1$.
    \end{enumerate}
  \end{problema}
 \begin{problema}
    Consideremos el siguiente producto infinito para el cálculo de $\pi$:
    (publicado en 1593 por François Viète y conocido como fórmula de Viète):
    \begin{equation*}
      \frac{2}{\pi}  = \prod_{k\to\infty} \frac{a_k}2, \qquad\text{donde }\quad
      a_0 = \sqrt{2}, \quad a_{k+1} = \sqrt{2+a_k}.
    \end{equation*}

    \begin{enumerate}
    \item Calcular las $15$ primeras aproximaciones de $\pi$ para la
      fórmula de Viète (correspondientes a la evaluación del producto
      anterior para $k=0,...,14$) y sus errores absolutos, $e_k$.
    \item Mostrar en una tabla los logaritmos de los errores, $\log
      e_k$, y los cocientes $\log e_{k+1}/\log
      e_{k}$ (con $k\ge 0$).
    \item Justificar la siguiente afirmación y utilizarla para estimar
      el orden de la fórmula de Viète:
      \begin{equation*}
        \label{eq:1}
        \{x_k\} \text{ es un método iterativo de orden } p 
        \Leftrightarrow \lim_{k\to\infty} \frac{\log e_{k+1}}{\log e_{k}}=p.
      \end{equation*}
    \end{enumerate}
  \end{problema}
  \begin{problema}
    Consideremos el problema de punto fijo $x=g(x)$, con
    $g(x)=1/(2+x)$.  Probar que existe una única solución en el
    intervalo $[0,1]$ y que método de aproximaciones sucesivas
    $x_{k+1}=g(x_k)$ converge hacia una solución para cualquier
    inicialización, $x_0\in[0,1]$. Aproximar la
    solución con un error menor que $\varepsilon=10^{-8}$.
  \end{problema}
  \begin{problema}
    Dada la ecuación $f(x)=e^x-(x+1)^2=0$:
    \begin{itemize}
    \item Estudiar sus raíces reales
    \item Localizarlas en intervalos en los que el método de Newton
      esté bien definido y sea convergente (idea: en cada intervalo,
      aplicar la regla de Fourier)
    \item En cada uno de estos intervalos, calcular una aproximación
      de la raíz de forma que el error con la solución exacta sea
      menor que $10^{-6}$.
    \end{itemize}
  \end{problema}
  \begin{problema}
    Calcular mediante el método de Newton las raices de
    $$f(x)=2\sqrt{x}-\frac{x^3+5}{5}$$ en $[0.1,1]$ realizando
    iteraciones hasta que la diferencia en dos etapas consecutivas sea
    menor que $10^{-10}$. ¿Converge el método si partes del punto
    medio del intervalo? Justifícalo geométricamente. ¿Y desde los
    extremos? Justifícalo mediante la Regla de Fourier.
  \end{problema}
  \begin{problema}
    Dada la ecuación $f(x)=e^x-(x-2)^2=0$:
    \begin{itemize}
    \item Justificar geométricamente que sólo posee una raíz real
    \item Demostrarlo analíticamente. Idea: probar que $f'(x)>0$ para
      todo $x\in\Rset$. Para ello, cuando $x>2$, aplicar a $e^x$ el
      teorema del valor medio en $[2,x]$ para probar que $e^x>2(x-2)$.
    \end{itemize}
  \end{problema}
  \begin{problema}
    Calcular las tres primeras aproximaciones ($k=0,1,2$) de $\pi$
    mediante el algoritmo de Gauss-Legendre (o de Brent-Salamin):
    $a_0=1$, $b_0=1/\sqrt 2$, $t_0=1/4$, $p_0=1$,
    \begin{align*}
      a_{k+1}&=(a_{k}+b_{k})/2 , &\quad
      t_{k+1}&=t_k - p_k(a_k -a_{k+1})^2, \\
      b_{k+1}& =\sqrt{a_{k}\*b_{k}}, &\quad p_{k+1}&=2p_k.
    \end{align*}
    $$
    \pi = \lim_{k\to\infty} \frac{(a_k+b_k)^2}{4t_k}.
    $$
    Utilizando los logaritmos de los errores absolutos, estimar el
    orden del algoritmo.
  \end{problema}
  \begin{problema}
    Consideremos el método de bisección para
    una función $f$ en el intervalo $[a,b]=[2.6,4.2]$. 
    \begin{enumerate}
    \item ¿Cuál es la longitud del subintervalo en el paso $k$--ésimo?
    \item ¿Cuál es la máxima distancia entre la raíz, $\alpha$, de $f$
      y el punto medio del intervalo en ese paso $k$--ésimo?
    \item ¿Cuántas iteraciones son necesarias, como mínimo, para que
      la longitud del intervalo sea menor que $\varepsilon=10^{-5}$?
    \end{enumerate}
  \end{problema}
\end{problemas}
\end{document}

%%% Local Variables: 
%%% mode: latex
%%% TeX-master: t
%%% End: 
