\documentclass[11pt]{article}
\usepackage[spanish]{babel}
\usepackage[utf8]{inputenc}

% ------ Cargar estilo específico para la relación de problemas
\usepackage{problemas-MNII}

% ==============
\begin{document}
% =============
\begin{flushright}
  \LARGE\it Relación de problemas. Tema \huge 3.\\
  \bigskip
\end{flushright}

\begin{problemas}

  \begin{problema}
    Consideremos el siguiente problema de valor inicial:
    \begin{align*}
      &y'=\frac{2y}{x} + x^2e^x, \quad x\in[1,2],\\
      &y(1)=0.
    \end{align*}
    Aproximar la solución con $h=0.1$ usando:
    \begin{enumerate}
    \item El método de Euler.
    \item El método de Euler--Cauchy (o Euler modificado).
    \end{enumerate}
    Comparar, en ambos casos, con la solución aproximada que
    proporciona Python (función \texttt{odeint} en
    \texttt{scipy.integrate}).
  \end{problema}
  

  \begin{problema}
    
  \end{problema}
  
  \end{problemas}
\end{document}

%%% Local Variables: 
%%% mode: latex
%%% TeX-master: t
%%% End: 
