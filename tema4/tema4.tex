\renewcommand{\tt}{t}
\newcommand{\yy}{y}
\newcommand{\ta}{a}
\newcommand{\tb}{b}
\newcommand{\ycero}{y_a}
\newcommand{\ee}{e\xspace}
\newcommand{\sol}{y}

\chapter[Problemas de valor inicial para EDOs de primer orden]
{Problemas de valor inicial para ecuaciones diferenciales de primer
  orden%
  \footnote{\licenseInfo}}

Son muy numerosos los problemas (con orígenes diversos como la física,
química, biología o economía) que se formulan matemáticamente en
términos de ecuaciones diferenciales, es decir, ecuaciones en las que
la incógnita es una función $y(x)$ que describe un fenómeno dado, formulado 
de sus derivadas (y de una serie de datos conocidos que garantizan el buen
planteamiento del problema). 

Este tema se centra en la resolución numérica de problemas de valor
inicial (o problemas de Cauchy) asociados a ecuaciones diferenciales
ordinarias (EDO). Este tipo de problemas se escriben de la siguiente
forma:
\begin{equation}
  \label{eq:pvi}
  \tag{PVI}
  \left\{
  \begin{aligned}
    &y' = f(\tt,\yy), \quad \tt\in[\ta,\tb],
    \\
    &y(\ta) = \ycero,
  \end{aligned}
  \right.
\end{equation}
donde $f:[\ta,\tb]\times\Rset\to\Rset$ \ee $\ycero\in\Rset$ son datos
conocidos. En el próximo tema se estudia el caso general, donde
$f:[\ta,\tb]\times\Rset^n\to\Rset^n$ \ee $\ycero\in\Rset^n$, con
$n\in\Nset$.

\begin{definition}
  Llamamos solución de~\eqref{eq:pvi} a toda función $\sol\in
  C^1([\ta,\tb])$ tal que $\sol'(\tt)=f(\tt,\sol(\tt))$ para todo
  $\tt\in[\ta,\tb]$ y además $\sol(\ta)=\ycero$.\label{def:3}
\end{definition}


Como paso previo a la resolución numérica de~\eqref{eq:pvi}, deberemos
garantizar que el problema~\eqref{eq:pvi} está bien planteado, es
decir asegurar la existencia y unicidad de solución en el sentido de
la definición anterior. Obsérvese que, según esta definición, el
concepto de solución tiene un sentido global (en todo el intervalo
$[a,b]$), no local (en un entorno de $\ta$).

Nuestro objetivo será la descripción, análisis e implementación de
distintos métodos numéricos para la aproximación de la solución
de~\eqref{eq:pvi}. Estos métodos partirán con la definición de una
partición del intervalo $[\ta,\tb]$, formada por $N+1$ puntos:
\begin{equation*}
  \ta=\tt_0 < \tt_1 < \cdots < \tt_N=\tb.
\end{equation*}
Por simplicidad, supondremos que la partición es uniforme, es decir,
\begin{equation*}
  \text{ si } h=\frac{b-a}{N}, \quad \text{entonces} \quad
  \tt_n=\ta+nh,\quad \forall n=0\dots,N
\end{equation*}
(en particular, $\tt_0=a$ y $\tt_N=b$). A continuación procedemos como
sigue:
\begin{enumerate}
\item Usamos el dato inicial $y_a$ para arrancar el método numérico,
  definiendo $y_0=y_a$ (más precisamente, $y_0\approx y_a$)
\item Seguidamente definimos una sucesión de forma que $y_{n+1}$ se
  calcula a partir de $y_n$ (y posiblemente de $y_{n-1}$, ...,
  $y_{n-k}$ para algún $k\ge 1$), con el fin de que
  $y_n\approx\sol(\tt_n)$. En concreto, de manera que:
  \begin{enumerate}
  \item El método sea \textit{consistente}, en el sentido de que la
    solución exacta $\{\sol(\tt_n)\}_n$ ``verifique aproximadamente'' el
    esquema numérico
  \item El método sea \textit{convergente}, en el sentido de que la
    solución aproximada $\{y_n\}_n$ ``converja a la solución exacta''
    $\{\sol(\tt_n)\}_n$ cuando $h\to 0$.
  \item El método sea \textit{estable}, en el sentido de que ``responda
    continuamente'' a perturbaciones en los datos iniciales.
  \end{enumerate}
\end{enumerate}
% Más adelante distinguiremos entre métodos de un paso, en los que
% para calcular $y_{n+1}$ utilizamos solamente de $y_n$ y métodos
% multipaso, en los que  $y_{n+1}$ depende de los datos en varias etapas
% anteriores ($y_{n}$, $y_{n-1}$, ..., $y_{n-k}$ para algún $k\ge 1$.)

%%% Local Variables:
%%% mode: latex
%%% TeX-master: "../apuntes-MNII.tex"
%%% End: 
