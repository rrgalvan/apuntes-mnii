\renewcommand{\tt}{t}
\newcommand{\yy}{y}
\newcommand{\yn}{{\yy_n}}
\newcommand{\ynn}{{\yy_{n+1}}}
\newcommand{\ta}{a}
\newcommand{\tb}{b}
\newcommand{\tn}{{\tt_n}}
\newcommand{\tnn}{{\tt_{n+1}}}
\newcommand{\zn}{{z_n}}
\newcommand{\znn}{{z_{n+1}}}
\newcommand{\ycero}{{y_a}}
% \newcommand{\sol}{\varphi}
\newcommand{\sol}{y}
\newcommand{\lipschitz}{$y$--Lipschitz\xspace}
\newcommand{\locLipschitz}{localmente \lipschitz}
\newcommand{\globLipschitz}{\lipschitz}
\newcommand{\errCons}{{\cal E}}
\newcommand{\RK}{Runge--Kutta\xspace}
\newcommand{\AB}{Adams--Bashforth\xspace}
\newcommand{\AM}{Adams--Moulton\xspace}

En este tema comprobaremos cómo los métodos numéricos que han sido
utilizados para la resolución numérica de problemas de valor inicial
para ecuaciones diferenciales ordinarias pueden ser fácilmente
extendidos al caso de sistemas de ecuaciones diferenciales. En
particular, esto dará lugar a algoritmos para resolver numéricamente
problemas de valor inicial para ecuaciones diferenciales de orden
mayor que uno.

Por simplicidad, nos restringimos a sistemas de dos ecuaciones y a
ecuaciones diferenciales de orden dos, aunque todos los resultados son
generalizables a cualquier número de ecuaciones y a cualquier orden.

\section{Problemas de valor inicial para sistemas de ecuaciones de
  primer orden}
\label{sec:sistemas-ecuaciones-dif}

Consideremos el problema de valores iniciales para $t\in [a,b]$:
\begin{align*}
  &x' = f(t, x, y), \\
  &y' = g(t, x, y), \\
  &y(a) = y_a, \quad z(a)=z_a
\end{align*}

Una solución de este sistema consistirá en un par de funciones $x(t)$,
$y(t)$ de clase $C^1([a,b])$ que verifican las ecuaciones anteriores
para todo $t\in[a,b]$.
Como en los temas anteriores, tendríamos que empezar planteando
condiciones sobre las funciones $f$ y $g$ que nos garanticen la
existencia y unicidad de solución del problema anterior. Sin embargo,
no entraremos en detalles: basta decir que los resultados de
existencia y unicidad son similares a los estudiados en el tema
anterior (basados en la condición de Lipschitz para las funciones $f$
y $g$).

Admitiendo, por tanto, el buen planteamiento del problema anterior,
pasamos directamente al estudio de métodos numéricos para la
aproximación de la solución.
Definimos una partición del intervalo $[\ta,\tb]$, formada por
$N+1$ puntos:
\begin{equation*}
  \ta=\tt_0 < \tt_1 < \cdots < \tt_N=\tb,
\end{equation*}
que por simplicidad, supondremos uniforme, es decir,
$\tt_n=\ta+n\cdot h$ para todo $n=0\dots,N$, donde $h=(b-a)/N$.
Queremos calcular dos conjuntos de números $x_n$, $y_n$, con
$n=0,\dots,N$ de forma que
$$
x_n \approx x(\tn), \quad y_n \approx y(\tn).
$$

Como en el tema anterior, si utilizamos las aproximaciones
\begin{align*}
  x'(\tn)\approx \frac{x(\tnn)-x(\tn)}{t_{n+1}-t_n} \approx \frac{x_{n+1}-x_n}h,
  \\
  y'(\tn)\approx \frac{y(\tnn)-y(\tn)}{t_{n+1}-t_n} \approx \frac{y_{n+1}-y_n}h,
\end{align*}
llegamos al método de \textbf{Euler explícito}: partiendo $x_0=x_a$, $y_0=y_a$,
\begin{align*}
  x_{n+1} = x_n + h f(\tn,x_n,y_n),
  \\
  y_{n+1} = y_n + h g(\tn,x_n,y_n).
\end{align*}

El resto de los métodos numéricos estudiados se pueden generalizar con facilidad a sistemas de ecuaciones. Por ejemplo, el método de \textbf{Euler-Cauchy}:
\begin{align*}
  x_{n+1} = x_n + h
  f\big(t_{n+1/2},\; x_n+\frac h2 f(\tn,x_n,y_n),\; y_n+\frac h2 g(\tn,x_n,y_n) \big),
  \\\noalign{\medskip}
  y_{n+1} = y_n + h
  g\big(t_{n+1/2},\; x_n+\frac h2 f(\tn,x_n,y_n),\; y_n+\frac h2 g(\tn,x_n,y_n) \big),
\end{align*}

Método de \textbf{Euler mejorado} (o método de \textbf{Heunn}): ejercicio.
% \begin{align*}
%   x_{n+1} = x_n + \frac h2
%   f\big(t_{n+1/2},\; x_n+\frac h2 f(\tn,x_n,y_n),\; y_n+\frac h2 g(\tn,x_n,y_n) \big),
%   \\\noalign{\medskip}
%   y_{n+1} = y_n + \frac h2
%   g\big(t_{n+1/2},\; x_n+\frac h2 f(\tn,x_n,y_n),\; y_n+\frac h2 g(\tn,x_n,y_n) \big).
% \end{align*}

%%% Local Variables:
%%% mode: latex
%%% TeX-master: "../tema5.tex"
%%% End:
